\documentclass[12pt,a4paper,sans]{moderncv}   % possible options include font size ('10pt', '11pt' and '12pt'), paper size ('a4paper', 'letterpaper', 'a5paper', 'legalpaper', 'executivepaper' and 'landscape') and font family ('sans' and 'roman')
\usepackage[utf8]{inputenc}  
\moderncvtheme[blue]{classic}     
\nopagenumbers{}                             
\AtBeginDocument{\recomputelengths}
%\usepackage[french]{babel} 


%\renewcommand{\listitemsymbol}{-~}  % change the symbol for lists
% color options 'blue' (default), 'orange', 'green', 'red', 'purple', 'grey' and 'black'
%\renewcommand{\familydefault}{\sfdefault}    % to set the default font; use '\sfdefault' for the default sans serif font, '\rmdefault' for the default roman one, or any tex font name

% adjust the page margins
\usepackage[top=1.1cm, bottom=1.1cm, left=2cm, right=2cm]{geometry}
% Largeur de la colonne pour les dates
\setlength{\hintscolumnwidth}{2.9cm}

\setlength{\makecvtitlenamewidth}{9cm}      % for the 'classic' style, if you want to force the width allocated to your name and avoid line breaks. be careful though, the length is normally calculated to avoid any overlap with your personal info; use this at your own typographical risks...

% personal data
\firstname{Martin}
\familyname{ACCOU}
\title{AI/ML Engineer}                        
\address{}{18 Rue de Saint-Omer}{62570, Wizernes}   
\mobile{+33 7 51 63 51 88}                   
\email{martin.accou@icloud.com} 
\extrainfo{French - 24 years old}
\usepackage{marvosym}

\usepackage{fontawesome5}

\faGrin{ maccou-portfolio.netlify.app}
\faLinkedin{ linkedin.com/in/martin-accou}
\faGithub{ github.com/accoumar12}

\photo[64pt][0pt]{martin.png}                  % optional, remove / comment the line if not wanted; '64pt' is the height the picture must be resized to, 0.4pt is the thickness of the frame around it (put it to 0pt for no frame) and 'picture' is the name of the picture file

\AfterPreamble{\hypersetup{
  pdfauthor={Martin ACCOU},
  pdftitle={CV_ACCOU_Martin},
  urlcolor=blue,
}}

\begin{document}

\makecvtitle

\section{Work experience}
\cventry{2024, 6 months}{ML engineer}{\href{https://d3s.ai}{D3S}}{Grenoble}{France}{\begin{itemize}
    \item Implementation of a 3D mesh classifier for automatic labeling of Computer-Aided Designs (CAD), leveraging a customized transformer-based architecture.
    \item \href{https://maccou-blog.netlify.app/building-a-3-d-instance-segmentation-model-for-cad-designs/}{Implementation of a 3D instance segmentation model tailored for CAD designs.}
    \item Development of a software tool for easy labeling of triplets of CAD designs, used for \href{https://maccou-blog.netlify.app/building-a-3-d-similarity-model-for-cad-designs/}{the training of an unsupervised classification model based on triplet loss.}
    \item Optimized the Open Cascade linker of the 3D analytics library, written in C.
    \end{itemize}}
\cventry{2023, 6 months}{Research data scientist}{\href{https://reuniwatt.com/en}{Reuniwatt}}{Saint-Pierre}{France}{\begin{itemize}
    \item Post-process day-ahead irradiance forecasts: benchmark several promising models for time series (Kalman filter, GBM, RF, SVR, MLP, N-BEATS, LSTM).
    \item Introduction to the issues relative to production and Machine Learning Operations.
    \item Worked in an agile team, contributing to sprint planning and iterative development.
    \end{itemize}}
\cventry{2022, 6 months}{Digital simulation engineer}{\href{https://www.defense.gouv.fr/dga/dga-techniques-aeronautiques}{DGA Technique aéronautiques}}{Balma}{France}{\begin{itemize}
    \item Study of parachutes during a cargo dropping (FEM and CFD).
    \item Learn new skills in digital simulation by working on LS-DYNA.
\end{itemize}}
\section{Education}
\cventry{2020-2024}{Master of Science}{\href{https://www.isae-supaero.fr/en}{ISAE-SUPAERO (top 5 eng. school)}}{Toulouse}{France}{
\begin{itemize}
    \item $1^{st}$ and $2^{nd}$ year: general engineering courses in computer science, applied mathematics, thermodynamics,  physics and mechanics.
    \item $3^{rd}$ year: Major in data science and data engineering. Minor in complex systems modeling and simulation: advanced statistics, HPC (OpenMP, MPI, CUDA).
    % \item Elective courses : mathematics for space applications, general relativity and cosmology, geophysical flows.
 \end{itemize}}  % arguments 3 to 6 can be left empty
\cventry{2022}{Semester  Erasmus+}{\href{https://www.polimi.it/en}{Politecnico di Milano (top 3 eng. school)}}{Milan}{Italy}{\begin{itemize}
    \item Courses: Software development (Vue.js, Nuxt.js), reinforcement learning, space physics, aeroacoustics, multibody system dynamics.
\end{itemize}}
\cventry{2018-2020}{Preparatory classes}{\href{https://www.bginette.com}{Lycée Sainte-Geneviève (top 3 school)}}{Versailles}{France}{\begin{itemize}
    \item MPSI/MP: two-year undergraduate intensive course in mathematics and physics.
\end{itemize}}


\section{Skills}

\cvitem{Programming}{Python, Go, C, Rust, Svelte.}
\cvitem{Data tools}{PyTorch, TensorFlow, Sklearn, W\&B, SQL, Spark, Dask, Docker, Kubernetes.}
\cvitem{Languages}{French (native), English (C1, TOEFL iBT : 102/120), Spanish (B1), Italian (A2).}

\section{Interests}

\cvitem{Hobbies}{Drums, travelling, open-source, linux, cybersecurity.}
\cvitem{Sports}{Trail-running (ultramarathon), cycling, soccer (10 years in club).}

\cvitem{Associative}{
Creator of the RAID trails and race chief for 2 years.
The \href{https://www.raidisae.fr/}{RAID ISAE} is an event including mountain bike, trail-running and canoe with more than 500 competitors.
}
               
\end{document}

